%% Generic short report include by Mateusz Machalica
%
\documentclass[a4paper]{article}

\usepackage[T1]{fontenc}
% this needs to be loaded before babel-polish
\usepackage{amssymb}
\usepackage{amsmath}
\usepackage{amsthm}
\usepackage[english]{babel}
\usepackage[utf8]{inputenc}
\usepackage[margin=2cm]{geometry}
\usepackage{graphicx}
\usepackage{ifthen}
\usepackage{hyperref}
\usepackage{algorithm}
\usepackage{algpseudocode}
\usepackage{wrapfig}
\usepackage{tikz}
\usepackage{float}
\usepackage{siunitx}
\usepackage{caption}
\usepackage{placeins}

\renewcommand*{\arraystretch}{1.5}

\newcommand{\makeheader}[3]{
  \begin{center}
    \begin{tabular*}{\textwidth}{@{\extracolsep{\fill}}lr}
      #2 & \ifthenelse{\equal{#3}{}}{\today ~r.}{#3} \\[4pt]
      \multicolumn{2}{c}{{\Large #1}} \\[4pt]
      \hline
    \end{tabular*}
    \vspace{4pt}
  \end{center}
}

\sisetup{
  output-decimal-marker = {.},
  separate-uncertainty = true,
  output-open-uncertainty = (,
  output-close-uncertainty = ),
  uncertainty-separator = \pm
}

\newcommand{\floor}[1]{ \lfloor #1 \rfloor }
\newcommand{\ceil}[1]{ \lceil #1 \rceil }
\newcommand{\floorfrac}[2]{ \floor{\frac{#1}{#2}} }
\newcommand{\ceilfrac}[2]{ \ceil{\frac{#1}{#2}} }
\newcommand{\pair}[2]{ \langle #1, #2 \rangle }
\newcommand{\stirset}[2]{ \{ {#1 \atop #2} \} }
\newcommand{\stirperm}[2]{ [ {#1 \atop #2} ] }
\newcommand{\set}[1]{ \{ #1 \} }
\newcommand{\tuple}[1]{ \langle #1 \rangle }
\newcommand{\e}[1]{ \mathbb{E}\left[ #1 \right] }
\newcommand{\odd}[1]{ \operatorname{odd}\left( #1 \right) }
\newcommand{\sgn}{ \operatorname{sgn} }

\newtheorem{lemma}{Lemma}
\newtheorem{theorem}[lemma]{Theorem}
\newtheorem{remark}[lemma]{Remark}
\newtheorem{problem}[lemma]{Problem}
\newenvironment{solution}{\begin{proof}[Solution]}{\end{proof}}

\algblock{ForParallel}{EndForParallel}
\algrenewtext{ForParallel}[1]{\textbf{for} #1 \textbf{do in parallel}}
\algrenewtext{EndForParallel}{\textbf{end for}}

% vim: noet:sw=2:ts=2